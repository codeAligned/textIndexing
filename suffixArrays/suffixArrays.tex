\chapter{Suffix Arrays}

\begin{Definition}
  The \defi{suffix array}{Suffix Array} $SA$ of a string $T$ gives the sorted order of all suffixes of $T$. Element $SA[i]$ stores the index of the $i$-th lexicographically smallest suffix of $T$.
\end{Definition}

\begin{Example}
  Consider the string $T=$"abracadabrabarbara\$". Table~\ref{tbl:suffixArrayExample} shows the suffix array for $T$. Note that the dollar sign \$ is lexicographically smaller than any other character. This guarantees that suffixes which are a prefix of other suffixes appear first.
  \begin{table}[htbp]
    \centering
    \begin{tabular}{ccl}
      \toprule
      $i$  & $SA[i]$ & $T[SA[i]..n-1]$ \\
      \midrule
      $0$  & $18$    & \$ \\
      $1$  & $17$    & a\$ \\
      $2$  & $10$    & abarbara\$ \\
      $3$  & $7$     & abrabarbara\$ \\
      $4$  & $0$     & abracadabrabarbara\$ \\
      $5$  & $3$     & acadabrabarbara\$ \\
      $6$  & $5$     & adabrabarbara\$ \\
      $7$  & $15$    & ara\$ \\
      $8$  & $12$    & arbara\$ \\
      $9$  & $14$    & bara\$ \\
      $10$ & $11$    & barbara\$ \\
      $11$ & $8$     & brabarbara\$ \\
      $12$ & $1$     & bracadabrabarbara\$ \\
      $13$ & $4$     & cadabrabarbara\$ \\
      $14$ & $6$     & dabrabarbara\$ \\
      $15$ & $16$    & ra\$ \\
      $16$ & $9$     & rabarbara\$ \\
      $17$ & $2$     & racadabrabarbara\$ \\
      $18$ & $13$    & rbara\$ \\
      \bottomrule
    \end{tabular}
    \caption{The suffix array for "abracadabrabarbara\$".}
    \label{tbl:suffixArrayExample}
  \end{table}
\end{Example}

\begin{Lemma}
  We can search for all occurrences of a string $S$ in $T$ using the suffix array $SA$ over $T$ in time $\mathcal{O}(m\log n)$, where $n = \vert T \vert$ and $m = \vert S \vert$. This is knows as \defi{forward search}{Forward Search}.
\end{Lemma}

\begin{Proof}
  We find the first suffix in $SA$ greater or equal to $S$ and the first suffix in $T$ bigger than $S$. Assume these are at positions $i$ and $j$ with $i \leq j$. Then we have $j - i$ occurrences of $S$ in $T$ and the corresponding positions are stored in $SA[i], \ldots, SA[j-1]$. Both $i$ and~$j$ can be found using a binary search, since the suffix array is sorted. The binary search needs $\mathcal{O}(\log n)$ steps and each steps does a string comparison in $\mathcal{O}(m)$ time.
\end{Proof}

\begin{Example}
  Consider strings $T=$"abracadabrabarbara\$" and $S=$"bar". How often and where does~$S$ appear in $T$?

  Table~\ref{tbl:suffixArrayExample} shows the suffix array for $T$. By binary search we find $i = 9$ and $j = 11$. So~$T[SA[i]..n-1]$ is the smallest suffix greater or equal than "bar" and $T[SA[j]..n-1]$ is the smallest suffix greater than "bar". We see that "bar" appears $j-i=11-9=2$ times. The occurrences are at positions $SA[9]=14$ and $SA[10]=11$.
\end{Example}

The suffix array itself needs $n\log n$ bits of space. But to work with it, we additionally need to store the text itself, which needs another $n\log\sigma$ bits of space. In total this results in a space of $n\log n + n\log\sigma$ bits.

% TODO (pjungeblut): Describe FM-index and the necessary BWT.
% TODO (pjungeblut): Describe how a self index can be obtained from the FM-index.
