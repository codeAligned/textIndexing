\section{Self Index}

The suffix array together with the Burrows-Wheeler-Transformation allowed to count the occurrences of a pattern $S$ in a string $T$ in $\mathcal{O}(m\log\sigma)$. The original text was not needed at all. But when we want to print the actual occurrences, we need to access the text $T$ and therefore store it together with the index. We will now see, how we can code the text into the index.

\begin{Definition}
  A \defi{self index}{Self Index} is an index that allows fast pattern matching and efficient reconstruction of any substring of the original text.
\end{Definition}

\begin{Definition}
  Let $j = SA[i]$ be the starting position of the $i$-th smallest suffix. Then \defi{$LF[i]$}{LF-Mapping} is defined as the position of $(j-1) \mod n$ in the suffix array.
\end{Definition}

\begin{Theorem}
  $LF$ can be calculated from the Burrows-Wheeler-Transformation:
  \begin{align}
    LF[i] := C[BWT[i]] + \mathrm{rank}_{BWT[i]}(i, BWT)
    \label{eq:lfMapping}
  \end{align}
\end{Theorem}

\begin{Proof}
  We know that $BWT[i]$ is the character preceding the suffix at position $i$ in the suffix array. So the previous suffix must be in the continuous range of suffixes in the suffix array starting with $BWT[i]$. This range begins at index $C[BWT[i]]$. Then $\mathrm{rank}_{BWT[i]}(i, BWT)$ calculates the offset in this range by just counting how many suffixes also start with $BWT[i]$ and are lexicographically smaller.
\end{Proof}

\begin{Theorem}
  \label{thm:lfMapping}
  The Burrows-Wheeler-Transformation and the $LF$-array are enough information to decode the whole text $T$.
\end{Theorem}

\begin{Proof}
  We will proof this by induction.

  \textbf{Base:} We know that the last suffix is the dollar sign "\$" and that it is stored at index $0$ in the suffix array $SA$, because \$ is lexicographically smaller than all other characters of our alphabet.

  \textbf{Step:} Assume we know some character $c$ and the index $i$, where the suffix of our text starting at $c$ is in the suffix array. Then the Burrows-Wheeler-Transformation $BWT[i]$ already gives us the character preceding $c$. To be able to pull of the same trick again, we still need the position of the suffix preceding the one starting at $c$. This is just how $LF[i]$ was defined.
\end{Proof}

\begin{Definition}
  The \defi{$F$-array}{$F$-Array} contains the first characters of each suffix of text $T$ in the order they appear in the suffix array.
  \begin{align}
    F[i] := T[SA[i]]
  \end{align}
\end{Definition}

\begin{Definition}
  The inverse of $LF$ is called \defi{$\Psi$}{$\Psi$-Mapping} and maps $j = SA[i]$, the position of the $i$-th smallest suffix, to the position of $(j+1) \mod n$ in the suffix array.
\end{Definition}

\begin{Theorem}
  For a text $T$ we get
  \begin{align}
    \Psi[i] := \mathrm{select}_{F[i]}(\mathrm{rank}_{F[i]}(i, F), BWT)
    \text{.}
  \end{align}
\end{Theorem}

\begin{Proof}
  If $j = SA[i]$ is the position of the $i$-th lexicographically smallest suffix, then the suffix starting at position $j+1$ is preceded by $F[i]$. Therefore we know the Burrows-Wheeler-Transformation $BWT[\Psi[i]] = F[i]$. This is why we can calculate $\Psi$ by a $\mathrm{select}$-query on the $BWT$-array. The inner query $\mathrm{rank}_{F[i]}(i, F)$ determines which occurrence of $F[i]$ in the $BWT$-array we are interested in. If the suffix at position $i$ in the suffix array is the $k$-th smallest starting with $F[i]$, then we look for the $k$-th occurrence of $F[i]$ in the $BWT$-array.
\end{Proof}

Function $\Psi$ allows to reconstruct the text from the front in contrast to $LF$, which can reconstruct it from the back as described in Theorem~\ref{thm:lfMapping}.

\begin{Definition}
  The suffix array is a permutation of the integers in $[0,n]$. The inverse permutation $ISA$ is called the \defi{inverse suffix array}{Inverse Suffix Array}. We get $i = ISA[SA[i]]$.
\end{Definition}

\begin{Theorem}
  \label{thm:lfPsiViaSAISA}
  $LF$ and $\Psi$ can be expressed using only the suffix array $SA$ and the inverse suffix array~$ISA$.
  \begin{align}
    LF[i] &= ISA[(SA[i] - 1) \mod n] \\
    \Psi[i] &= ISA[(SA[i] + 1) \mod n]
  \end{align}
\end{Theorem}

\begin{Proof}
  $LF[i]$ was defined as the position of $(SA[i] - 1) \mod n$ in the suffix array. If we know the inverse permutation of the suffix array $ISA$, we can just look into it for the position we want. For $\Psi$ this works analogously.
\end{Proof}

\begin{Theorem}
  $SA$, $ISA$, $LF$ and $\Psi$ can be accessed using $n \log n + o(n \log n)$ bits space each in time $\mathcal{O}(\log n)$ .
\end{Theorem}

\begin{Proof}
  We can build a wavelet tree over the suffix array $SA$. By Theorem~\ref{thm:waveletTreeAceess} we can than access each element in the suffix array in time $\mathcal{O}(\log n)$. To access the inverse suffix array $ISA$ we would need the position of some value $i$ in the suffix array. This can be done with a single $\mathrm{select}$-query on the wavelet tree. By Theorem~\ref{thm:waveletTreeSelect} this takes time $\mathcal{O}(\log n)$ time as well. Last, we know from Theorem~\ref{thm:lfPsiViaSAISA} that efficient access to $SA$ and $ISA$ is enough to calculate $LF$ and $\Psi$.
\end{Proof}

\begin{Example}
  Table~\ref{tbl:suffixArraySummary} shows the suffix array for the string "abracadabrabarbara\$" together with all the different arrays introduced in this sectin.
  \begin{table}[htb]
    \centering
    \begin{tabular}{ccccccl}
      \toprule
      $i$&$SA[i]$&$LF[i]$&$\Psi[i]$&$BWT[i]$&$F[i]$& $T[SA[i]..n-1]$ \\
      \midrule
      $0$  & $18$ & $1$  & $4$  & a  & \$ & \$ \\
      $1$  & $17$ & $15$ & $0$  & r  & a  & a\$ \\
      $2$  & $10$ & $16$ & $10$ & r  & a  & abarbara\$ \\
      $3$  & $7$  & $14$ & $11$ & d  & a  & abrabarbara\$ \\
      $4$  & $0$  & $0$  & $12$ & \$ & a  & abracadabrabarbara\$ \\
      $5$  & $3$  & $17$ & $13$ & r  & a  & acadabrabarbara\$ \\
      $6$  & $5$  & $13$ & $14$ & c  & a  & adabrabarbara\$ \\
      $7$  & $15$ & $9$  & $15$ & b  & a  & ara\$ \\
      $8$  & $12$ & $10$ & $18$ & b  & a  & arbara\$ \\
      $9$  & $14$ & $18$ & $7$  & r  & b  & bara\$ \\
      $10$ & $11$ & $2$  & $8$  & a  & b  & barbara\$ \\
      $11$ & $8$  & $3$  & $16$ & a  & b  & brabarbara\$ \\
      $12$ & $1$  & $4$  & $17$ & a  & b  & bracadabrabarbara\$ \\
      $13$ & $4$  & $5$  & $6$  & a  & c  & cadabrabarbara\$ \\
      $14$ & $6$  & $6$  & $3$  & a  & d  & dabrabarbara\$ \\
      $15$ & $16$ & $7$  & $1$  & a  & r  & ra\$ \\
      $16$ & $9$  & $11$ & $2$  & b  & r  & rabarbara\$ \\
      $17$ & $2$  & $12$ & $5$  & b  & r  & racadabrabarbara\$ \\
      $18$ & $13$ & $8$  & $9$  & a  & r  & rbara\$ \\
      \bottomrule
    \end{tabular}
    \caption{The suffix array for "abracadabrabarbara\$".}
    \label{tbl:suffixArraySummary}
  \end{table}
\end{Example}

\begin{Theorem}
  \label{thm:psiValuesIncreasing}
  The values of $\Psi$ form at most $\sigma$ increasing integer sequences. Consider Table~\ref{tbl:suffixArraySummary} as an example.
\end{Theorem}

\begin{Proof}
  We can divide the entries in the suffix array into $\sigma$ continuous parts, grouping the suffixes starting with the same symbol $c \in \Sigma$. In each group the suffixes are sorted lexicographically. Because the first character in each group is the same for each suffix, their second characters form an increasing sequence. $\Psi$ maps to the positions of the suffixes starting at these second characters, so the $\Psi$ values are also increasing.
\end{Proof}
