\section{Self Index}

The suffix array together with the Burrows-Wheeler-Transformation allowed to count the occurrences of a pattern $S$ in a string $T$ in $\mathcal{O}(m\log\sigma)$. The original text was not needed at all. But when we want to print the actual occurrences, we need to access the text $T$ and therefore store it together with the index. We will now see, how we can code the text into the index.

\begin{Definition}
  A \defi{self index}{Self Index} is an index that allows fast pattern matching and efficient reconstruction of any substring of the original text.
\end{Definition}

\begin{Definition}
  Let $j = \id{SA}[i]$ be the starting position of the $i$-th smallest suffix. Then \defi{$\id{LF}[i]$}{\id{LF}-Mapping} is defined as the position of $(j-1) \mod n$ in the suffix array.
\end{Definition}

\begin{Theorem}
  \id{LF} can be calculated from the Burrows-Wheeler-Transformation:
  \begin{align}
    \id{LF}[i] := C[\id{BWT}[i]] + \proc{Rank}_{\id{BWT}[i]}(i, \id{BWT})
    \label{eq:lfMapping}
  \end{align}
\end{Theorem}

\begin{Proof}
  We know that $\id{BWT}[i]$ is the character preceding the suffix at position $i$ in the suffix array. So the previous suffix must be in the continuous range of suffixes in the suffix array starting with $\id{BWT}[i]$. This range begins at index $C[\id{BWT}[i]]$. Then $\proc{Rank}_{\id{BWT}[i]}(i, \id{BWT})$ calculates the offset in this range by just counting how many suffixes also start with $\id{BWT}[i]$ and are lexicographically smaller.
\end{Proof}

\begin{Theorem}
  \label{thm:lfMapping}
  The Burrows-Wheeler-Transformation and the \id{LF}-array are enough information to decode the whole text $T$.
\end{Theorem}

\begin{Proof}
  We will proof this by induction.

  \textbf{Base:} We know that the last suffix is the dollar sign \texttt{\$} and that it is stored at index $0$ in the suffix array \id{SA}, because \texttt{\$} is lexicographically smaller than all other characters of our alphabet.

  \textbf{Step:} Assume we know some character $c$ and the index $i$, where the suffix of our text starting at $c$ is in the suffix array. Then the Burrows-Wheeler-Transformation $\id{BWT}[i]$ already gives us the character preceding $c$. To be able to pull of the same trick again, we still need the position of the suffix preceding the one starting at $c$. This is just how $\id{LF}[i]$ was defined.
\end{Proof}

\begin{Definition}
  The \defi{$F$-array}{$F$-Array} contains the first characters of each suffix of text $T$ in the order they appear in the suffix array.
  \begin{align}
    F[i] := T[\id{SA}[i]]
  \end{align}
\end{Definition}

\begin{Definition}
  The inverse of \id{LF} is called \defi{$\Psi$}{$\Psi$-Mapping} and maps $j = \id{SA}[i]$, the position of the $i$-th smallest suffix, to the position of $(j+1) \mod n$ in the suffix array.
\end{Definition}

\begin{Theorem}
  For a text $T$ we get
  \begin{align}
    \Psi[i] := \proc{Select}_{F[i]}(\proc{Rank}_{F[i]}(i, F), \id{BWT})
    \text{.}
  \end{align}
\end{Theorem}

\begin{Proof}
  If $j = \id{SA}[i]$ is the position of the $i$-th lexicographically smallest suffix, then the suffix starting at position $j+1$ is preceded by $F[i]$. Therefore we know the Burrows-Wheeler-Transformation $\id{BWT}[\Psi[i]] = F[i]$. This is why we can calculate $\Psi$ by a $\proc{Select}$-query on the \id{BWT}-array. The inner query $\proc{Rank}_{F[i]}(i, F)$ determines which occurrence of $F[i]$ in the \id{BWT}-array we are interested in. If the suffix at position $i$ in the suffix array is the $k$-th smallest starting with $F[i]$, then we look for the $k$-th occurrence of $F[i]$ in the \id{BWT}-array.
\end{Proof}

Function $\Psi$ allows to reconstruct the text from the front in contrast to \id{LF}, which can reconstruct it from the back as described in Theorem~\ref{thm:lfMapping}.

\begin{Definition}
  The suffix array is a permutation of the integers in $[0,n]$. The inverse permutation \id{ISA} is called the \defi{inverse suffix array}{Inverse Suffix Array}. We get $i = \id{ISA}[\id{SA}[i]]$.
\end{Definition}

\begin{Theorem}
  \label{thm:lfPsiViaSAISA}
  \id{LF} and $\Psi$ can be expressed using only the suffix array \id{SA} and the inverse suffix array~\id{ISA}.
  \begin{align}
    \id{LF}[i] &= \id{ISA}[(\id{SA}[i] - 1) \mod n] \\
    \Psi[i] &= \id{ISA}[(\id{SA}[i] + 1) \mod n]
  \end{align}
\end{Theorem}

\begin{Proof}
  $\id{LF}[i]$ was defined as the position of $(\id{SA}[i] - 1) \mod n$ in the suffix array. If we know the inverse permutation of the suffix array \id{ISA}, we can just look into it for the position we want. For $\Psi$ this works analogously.
\end{Proof}

\begin{Theorem}
  \id{SA}, \id{ISA}, \id{LF} and $\Psi$ can be accessed using $n \log n + o(n \log n)$ bits space each in time $\mathcal{O}(\log n)$ .
\end{Theorem}

\begin{Proof}
  We can build a wavelet tree over the suffix array \id{SA}. By Theorem~\ref{thm:waveletTreeAceess} we can than access each element in the suffix array in time $\mathcal{O}(\log n)$. To access the inverse suffix array \id{ISA} we would need the position of some value $i$ in the suffix array. This can be done with a single \proc{Select}-query on the wavelet tree. By Theorem~\ref{thm:waveletTreeSelect} this takes time $\mathcal{O}(\log n)$ time as well. Last, we know from Theorem~\ref{thm:lfPsiViaSAISA} that efficient access to \id{SA} and \id{ISA} is enough to calculate \id{LF} and $\Psi$.
\end{Proof}

\begin{Example}
  Table~\ref{tbl:suffixArraySummary} shows the suffix array for the string $T=\texttt{abracadabrabarbara\$}$ together with all the different arrays introduced in this section.
  \begin{table}[htb]
    \centering
    \begin{tabular}{ccccccl}
      \toprule
      $i$ & $\id{SA}[i]$ & $\id{LF}[i]$ & $\Psi[i]$ & $\id{BWT}[i]$ & $F[i]$ & $T[\id{SA}[i] \twodots n-1]$ \\
      \midrule
      $0$  & $18$ & $1$  & $4$  & \texttt{a}  & \texttt{\$} & \texttt{\$} \\
      $1$  & $17$ & $15$ & $0$  & \texttt{r}  & \texttt{a}  & \texttt{a\$} \\
      $2$  & $10$ & $16$ & $10$ & \texttt{r}  & \texttt{a}  & \texttt{abarbara\$} \\
      $3$  & $7$  & $14$ & $11$ & \texttt{d}  & \texttt{a}  & \texttt{abrabarbara\$} \\
      $4$  & $0$  & $0$  & $12$ & \texttt{\$} & \texttt{a}  & \texttt{abracadabrabarbara\$} \\
      $5$  & $3$  & $17$ & $13$ & \texttt{r}  & \texttt{a}  & \texttt{acadabrabarbara\$} \\
      $6$  & $5$  & $13$ & $14$ & \texttt{c}  & \texttt{a}  & \texttt{adabrabarbara\$} \\
      $7$  & $15$ & $9$  & $15$ & \texttt{b}  & \texttt{a}  & \texttt{ara\$} \\
      $8$  & $12$ & $10$ & $18$ & \texttt{b}  & \texttt{a}  & \texttt{arbara\$} \\
      $9$  & $14$ & $18$ & $7$  & \texttt{r}  & \texttt{b}  & \texttt{bara\$} \\
      $10$ & $11$ & $2$  & $8$  & \texttt{a}  & \texttt{b}  & \texttt{barbara\$} \\
      $11$ & $8$  & $3$  & $16$ & \texttt{a}  & \texttt{b}  & \texttt{brabarbara\$} \\
      $12$ & $1$  & $4$  & $17$ & \texttt{a}  & \texttt{b}  & \texttt{bracadabrabarbara\$} \\
      $13$ & $4$  & $5$  & $6$  & \texttt{a}  & \texttt{c}  & \texttt{cadabrabarbara\$} \\
      $14$ & $6$  & $6$  & $3$  & \texttt{a}  & \texttt{d}  & \texttt{dabrabarbara\$} \\
      $15$ & $16$ & $7$  & $1$  & \texttt{a}  & \texttt{r}  & \texttt{ra\$} \\
      $16$ & $9$  & $11$ & $2$  & \texttt{b}  & \texttt{r}  & \texttt{rabarbara\$} \\
      $17$ & $2$  & $12$ & $5$  & \texttt{b}  & \texttt{r}  & \texttt{racadabrabarbara\$} \\
      $18$ & $13$ & $8$  & $9$  & \texttt{a}  & \texttt{r}  & \texttt{rbara\$} \\
      \bottomrule
    \end{tabular}
    \caption{The suffix array for $T=\texttt{abracadabrabarbara\$}$.}
    \label{tbl:suffixArraySummary}
  \end{table}
\end{Example}

\begin{Theorem}
  \label{thm:psiValuesIncreasing}
  The values of $\Psi$ form at most $\sigma$ increasing integer sequences. Consider Table~\ref{tbl:suffixArraySummary} as an example.
\end{Theorem}

\begin{Proof}
  We can divide the entries in the suffix array into $\sigma$ continuous parts, grouping the suffixes starting with the same symbol $c \in \Sigma$. In each group the suffixes are sorted lexicographically. Because the first character in each group is the same for each suffix, their second characters form an increasing sequence. $\Psi$ maps to the positions of the suffixes starting at these second characters, so the $\Psi$ values are also increasing.
\end{Proof}
