\section{Sampling}

Instead of storing the complete suffix array we can only store a subset of the values to save space.

\begin{Definition}
  Fix a sampling parameter $s$ and add a bitvector $B$ of length $n$ with $B[i] = 1$ if and only if $\id{SA}[i] \equiv 0 \mod s$. Store exactly those elements of \id{SA} in another array \id{SA'} of size $\lceil \frac{n}{s} \rceil$, the \defi{sampled suffix array}{Sampled Suffix Array}. For all $i$ with $B[i] = 1$, we get
  \begin{align}
    \id{SA'}[\proc{Rank}_1(i, B)] := \id{SA}[i]
    \text{.}
  \end{align}
  This is known as \defi{SA-order-sampling}{SA-Order-Sampling}.
\end{Definition}

\begin{Definition}
  Fix a sampling parameter $s$. Store exactly those elements $\id{SA}[i]$ with $i \equiv 0 \mod s$ in another array \id{SA'} of size $\lceil \frac{n}{s} \rceil$. We get
  \begin{align}
    \id{SA'}[i/s] := \id{SA}[i]
    \text{.}
  \end{align}
  This is known as \defi{text-order-sampling}{Text-Order-Sampling}.
\end{Definition}

No matter which of the above sampling strategies above is used, the code in Algorithm~\ref{alg:sampledSuffixArrayAccess} allows to access arbitrary elements of the suffix array.

\begin{algorithm}[htb]
  \begin{codebox}
  \Procname{$\proc{Access-Sampled-Suffix-Array}(i)$}
  \li $k \gets 0$
  \li \While $B[i] \isequal 0$
      \Do
  \li   $k \gets k + 1$
  \li   $i \gets \id{LF}[i]$    
      \End
  \li \Return $\id{SA'}[\proc{Rank}_1(i, B)] + k$
  \end{codebox}
  \caption{Access the sampled suffix array.}
  \label{alg:sampledSuffixArrayAccess}
\end{algorithm}

\begin{Theorem}
  \label{thm:sampledSuffixArrayAccess}
  If SA-order-sampling is used, the $\proc{Access-Sampled-Suffix-Array}$-method in Algorithm~\ref{alg:sampledSuffixArrayAccess} takes at most $\mathcal{O}(s\cdot t_{\id{LF}})$ time, where $t_{\id{LF}}$ is the time for an access in the \id{LF} array.
\end{Theorem}

\begin{Proof}
  Assume we want to access index $i$ of the suffix array \id{SA} and let $j = \id{SA}[i]$. Each $i \gets \id{LF}[i]$ call executed in Algorithm~\ref{alg:sampledSuffixArrayAccess} $j$ becomes $j-1$, so after at most $s - 1$ iterations $j - k = \id{SA}[i]$ is divisible by $s$ and therefore sampled. We can return $j - k + k = j$.
\end{Proof}

\begin{Theorem}
  SA-order-sampling as described needs at most $\frac{n}{s}\log n + 2n$ bits space. This can be improved to $\frac{n}{s}\log\frac{n}{s} + 2n$ bits, with the $\proc{Access-Sampled-Suffix-Array}$-operation still needing time in $\mathcal{O}(s\cdot t_{\id{LF}})$.
\end{Theorem}

\begin{Proof}
  The sampled suffix array has $\frac{n}{s}$ elements, each an integer from $[0,n-1]$. The additional $2n$ bits are $n$-bit bitvector $B$ and the overhead to allow $\proc{Rank}$-operations in constant time.

  The necessary observation to reduce the space needed is that each sampled element $\id{SA}[i]$ is a multiple of the sampling parameter $s$. So instead of \begin{align}
    \id{SA'}[\proc{Rank}_1(i, B)] := \id{SA}[i]
  \end{align}
  we can write
  \begin{align}
    \id{SA'}[\proc{Rank}_1(i, B)] := \frac{\id{SA}[i]}{s}
    \text{.}
  \end{align}
  The access operation stays the same, instead of the return call in the last line: We need to multiply the value read from \id{SA'} by $s$.
\end{Proof}

\begin{Theorem}
  The inverse suffix array \id{ISA} can be sampled in $\frac{n}{s}\log n$ bits space using an array with $\frac{n}{s}$ elements and allowing to access arbitrary elements of \id{ISA} in time $\mathcal{O}(s\cdot t_{\id{LF}})$.
\end{Theorem}

\begin{Proof}
  $\id{ISA}[i]$ gives the lexicographical rank of the $i$-th suffix. Therefore
  \begin{align}
    \id{LF}[\id{ISA}[i]] = \id{ISA}[(i-1) \mod n]
    \text{.}
  \end{align}
  By induction we get
  \begin{align}
    \id{LF}^k[\id{ISA}[i]] = \id{ISA}[(i - k) \mod n]
    \text{.}
  \end{align}
  We now choose a sampling parameter $s$ and allocate an array \id{ISA'} with $\frac{n}{s}$ elements. For every $i \equiv 0 \mod s$ we store
  \begin{align}
    \id{ISA'}[i/s] = \id{ISA}[i]
    \text{.}
  \end{align}
  To access $\id{ISA}[i]$ for some $i$ we can use the code from Algorithm~\ref{alg:sampledInverseSuffixArrayAccess}.
  \begin{algorithm}[htb]
    \begin{codebox}
      \Procname{$\proc{Access-Sampled-Inverse-Suffix-Array}(i)$}
      \li $\id{idx} \gets \left \lceil \frac{i}{s} \right \rceil$
      \li $\id{diff} \gets \id{idx} - i$
      \li \Return $\id{LF}^{\id{diff}}[\id{ISA'}[\id{idx}]]$
    \end{codebox}
    \caption{Access the sampled inverse suffix array.}
    \label{alg:sampledInverseSuffixArrayAccess}
  \end{algorithm}
\end{Proof}

The bitvector storing the sampled positions of the the suffix array when using SA-order-sampling is a considerable space overhead for texts over a small alphabet. This overhead is saved when using text-order-sampling. However the upper bound of at most $s-1$ \id{LF}-calls does not hold any more. Consider a text like \texttt{abcdeabcde\$} that consists of two copies of the same string and a sampling parameter of $s=2$. Table~\ref{tbl:suffixArraySamplingComparison} shows the corresponding suffix array. The stars indicate which elements are sampled ($\id{SA}[i] \equiv 0$ for SA-order-sampling, $i \equiv 0$ for text-order-sampling).

\begin{table}[htb]
  \centering
  \begin{tabular}{ccccl}
    \toprule
    $i$  & $\id{SA}[i]$ & $\id{SA}[i] \equiv 0$ & $i \equiv 0$ & $T[i \twodots n-1]$ \\
    \midrule
    $0$  & $10$    & \texttt{*} & \texttt{*} & \texttt{\$} \\
    $1$  & $5$     &            &            & \texttt{abcde\$} \\
    $2$  & $0$     & \texttt{*} & \texttt{*} & \texttt{abcdeabcde\$} \\
    $3$  & $6$     & \texttt{*} &            & \texttt{bcde\$} \\
    $4$  & $1$     &            & \texttt{*} & \texttt{bcdeabcde\$} \\
    $5$  & $7$     &            &            & \texttt{cde\$} \\
    $6$  & $2$     & \texttt{*} & \texttt{*} & \texttt{cdeabcde\$} \\
    $7$  & $8$     & \texttt{*} &            & \texttt{de\$} \\
    $8$  & $3$     &            & \texttt{*} & \texttt{deabcde\$} \\
    $9$  & $9$     &            &            & \texttt{e\$} \\
    $10$ & $4$     & \texttt{*} & \texttt{*} & \texttt{eabcde\$} \\
    \bottomrule
  \end{tabular}
  \caption{The suffix array of $T=\texttt{abcdeabcde\$}$ with different sampling positions. The sampling parameter is $s=2$.}
  \label{tbl:suffixArraySamplingComparison}
\end{table}

Imagine we want to access $\id{SA}[9]$. Both sampling strategies did not sample this value. SA-order-sampling needs at most $s-1=1$ \id{LF}-call. Text-order-sampling however needs $5$ \id{LF}-calls. In general for a text like this of length $n$, $\frac{n}{s}$ \id{LF}-calls are necessary in the worst case, which is linear in the text length.
