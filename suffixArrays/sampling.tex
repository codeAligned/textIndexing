\section{Sampling}

Instead of storing the complete suffix array we can only store a subset of the values to save space.

\begin{Definition}
  Fix a sampling parameter $s$ and add a bitvector $B$ of length $n$ with $B[i] = 1$ if and only if $SA[i] \equiv 0 \mod s$. Store exactly those elements of $SA$ in another array $SA'$ of size $\lceil \frac{n}{s} \rceil$, the \defi{sampled suffix array}{Sampled Suffix Array}. For all $i$ with $B[i] = 1$, we get
  \begin{align}
    SA'[\mathrm{rank}_1(i, B)] := SA[i]
    \text{.}
  \end{align}
  This is known as \defi{SA-order-sampling}{SA-Order-Sampling}.
\end{Definition}

\begin{Definition}
  Fix a sampling parameter $s$. Store exactly those elements $SA[i]$ with $i \equiv 0 \mod s$ in another array $SA'$ of size $\lceil \frac{n}{s} \rceil$. We get
  \begin{align}
    SA'[i/s] := SA[i]
    \text{.}
  \end{align}
  This is known as \defi{text-order-sampling}{Text-Order-Sampling}.
\end{Definition}

No matter which of the above sampling strategies above is used, the code in Algorithm~\ref{alg:sampledSuffixArrayAccess} allows to access arbitrary elements of the suffix array.

\begin{pseudocode}
  {$\mathrm{access}$}
  {sampledSuffixArrayAccess}
  {Index $i$ to access.}
  {Value of $SA[i]$.}
  \STATE $k \gets 0$
  \WHILE{$B[i] = 0$}
    \STATE $k \gets k + 1$
    \STATE $i \gets LF[i]$
  \ENDWHILE
  \RETURN $SA'[rank_1(i, B)] + k$
\end{pseudocode}

\begin{Theorem}
  \label{thm:sampledSuffixArrayAccess}
  If SA-order-sampling is used, the $\mathrm{access}$-method in Algorithm~\ref{alg:sampledSuffixArrayAccess} takes at most $\mathcal{O}(s\cdot t_{LF})$ time, where $t_{LF}$ is the time for an access in the $LF$ array.
\end{Theorem}

\begin{Proof}
  Assume we want to access index $i$ of the suffix array $SA$ and let $j = SA[i]$. Each $i \leftarrow LF[i]$ call executed in Algorithm~\ref{alg:sampledSuffixArrayAccess} $j$ becomes $j-1$, so after at most $s - 1$ iterations $j - k = SA[i]$ is divisible by $s$ and therefore sampled. We can return $j - k + k = j$.
\end{Proof}

\begin{Theorem}
  SA-order-sampling as described needs at most $\frac{n}{s}\log n + 2n$ bits space. This can be improved to $\frac{n}{s}\log\frac{n}{s} + 2n$ bits, with the $\mathrm{access}$-operation still needing time in $\mathcal{O}(s\cdot t_{LF})$.
\end{Theorem}

\begin{Proof}
  The sampled suffix array has $\frac{n}{s}$ elements, each an integer from $[0,n-1]$. The additional $2n$ bits are $n$-bit bitvector $B$ and the overhead to allow $\mathrm{rank}$-operations in constant time.

  The necessary observation to reduce the space needed is that each sampled element $SA[i]$ is a multiple of the sampling parameter $s$. So instead of \begin{align}
    SA'[\mathrm{rank}_1(i, B)] := SA[i]
  \end{align}
  we can write
  \begin{align}
    SA'[\mathrm{rank}_1(i, B)] := \frac{SA[i]}{s}
    \text{.}
  \end{align}
  The access operation stays the same, instead of the return call in the last line: We need to multiply the value read from $SA'$ by $s$.
\end{Proof}

\begin{Theorem}
  The inverse suffix array $ISA$ can be sampled in $\frac{n}{s}\log n$ bits space using an array with $\frac{n}{s}$ elements and allowing to access arbitrary elements of $ISA$ in time $\mathcal{O}(s\cdot t_{LF})$.
\end{Theorem}

\begin{Proof}
  $ISA[i]$ gives the lexicographical rank of the $i$-th suffix. Therefore
  \begin{align}
    LF[ISA[i]] = ISA[(i-1) \mod n]
    \text{.}
  \end{align}
  By induction we get
  \begin{align}
    LF^k[ISA[i]] = ISA[(i - k) \mod n]
    \text{.}
  \end{align}
  We now choose a sampling parameter $s$ and allocate an array $ISA'$ with $\frac{n}{s}$ elements. For every $i \equiv 0 \mod s$ we store
  \begin{align}
    ISA'[i/s] = ISA[i]
    \text{.}
  \end{align}
  To access $ISA[i]$ for some $i$ we can use the code from Algorithm~\ref{alg:sampledInverseSuffixArrayAccess}.
  \begin{pseudocode}
    {$\mathrm{access}$}
    {sampledInverseSuffixArrayAccess}
    {Index $i$ to access.}
    {Value of $ISA[i]$.}
    \STATE $idx \gets \lceil \frac{i}{s} \rceil$
    \STATE $diff \gets idx - i$
    \RETURN $LF^{diff}[ISA'[idx]]$
  \end{pseudocode}
\end{Proof}

The bitvector storing the sampled positions of the the suffix array when using SA-order-sampling is a considerable space overhead for texts over a small alphabet. This overhead is saved when using text-order-sampling. However the upper bound of at most $s-1$ $LF$-calls does not hold any more. Consider a text like "abcdeabcde\$" that consists of two copies of the same string and a sampling parameter of $s=2$. Table~\ref{tbl:suffixArraySamplingComparison} shows the corresponding suffix array. The stars indicate which elements are sampled ($SA[i] \equiv 0$ for SA-order-sampling, $i \equiv 0$ for text-order-sampling).

\begin{table}[htbp]
  \centering
  \begin{tabular}{ccccl}
    \toprule
    $i$  & $SA[i]$ & $SA[i] \equiv 0$ & $i \equiv 0$ & $T[i..n-1]$ \\
    \midrule
    $0$  & $10$    & *                & *            & \$ \\
    $1$  & $5$     &                  &              & abcde\$ \\
    $2$  & $0$     & *                & *            & abcdeabcde\$ \\
    $3$  & $6$     & *                &              & bcde\$ \\
    $4$  & $1$     &                  & *            & bcdeabcde\$ \\
    $5$  & $7$     &                  &              & cde\$ \\
    $6$  & $2$     & *                & *            & cdeabcde\$ \\
    $7$  & $8$     & *                &              & de\$ \\
    $8$  & $3$     &                  & *            & deabcde\$ \\
    $9$  & $9$     &                  &              & e\$ \\
    $10$ & $4$     & *                & *            & eabcde\$ \\
    \bottomrule
  \end{tabular}
  \caption{The suffix array of $T=$"abcdeabcde\$" with different sampling positions. The sampling parameter is $s=2$.}
  \label{tbl:suffixArraySamplingComparison}
\end{table}

Imagine we want to access $SA[9]$. Both sampling strategies did not sample this value. SA-order-sampling needs at most $s-1=1$ $LF$-call. Text-order-sampling however needs $5$ $LF$-calls. In general for a text like this of length $n$, $\frac{n}{s}$ $LF$-calls are necessary in the worst case, which is linear in the text length.

\begin{Theorem}
  The values of $\Psi$ form at most $\sigma$ increasing integer sequences. Consider Table~\ref{tbl:suffixArraySummary} as an example.
\end{Theorem}

\begin{Proof}
  We can divide the entries in the suffix array into $\sigma$ continuous parts, grouping the suffixes starting with the same symbol $c \in \Sigma$. In each group the suffixes are sorted lexicographically. Because the first character in each group is the same for each suffix, their second characters form an increasing sequence. $\Psi$ maps to the positions of the suffixes starting at these second characters, so the $\Psi$ values are also increasing.
\end{Proof}
