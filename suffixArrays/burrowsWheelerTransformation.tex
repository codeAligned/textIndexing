\section{Burrows-Wheeler-Transformation}

\begin{Definition}
  The \defi{Burrows-Wheeler-Transformation}{Burrows-Wheeler-Transformation} $BWT$ rearranges the characters of a string $T$ into runs of similar characters which makes it easier to compress.
  \begin{equation}
    BWT[i] := T[SA[i] - 1 \mod n]
  \end{equation}
  Intuitively, $BWT[i]$ is the character preceding the $i$-th suffix (or the last character for the zeroth suffix).
\end{Definition}

The BWT needs $n\log\sigma$ bits of space if it is stored uncompressed. The idea behind the BWT was better compression: Compressed size is $nH_k(T)$ bits, where $H_k(T)$ is the $k-th$ order entropy of the text $T$.

\begin{Example}
  Table~\ref{tbl:burrowsWheelerTransformationExample} shows the suffix array for the string "abracadabrabarbara\$" together with the Burrows-Wheeler-Transformation of it.
  \begin{table}[htbp]
    \centering
    \begin{tabular}{cccl}
      \toprule
      $i$  & $SA[i]$ & $BWT[i]$ & $T[SA[i]..n-1]$ \\
      \midrule
      $0$  & $18$    & a        & \$ \\
      $1$  & $17$    & r        & a\$ \\
      $2$  & $10$    & r        & abarbara\$ \\
      $3$  & $7$     & d        & abrabarbara\$ \\
      $4$  & $0$     & \$       & abracadabrabarbara\$ \\
      $5$  & $3$     & r        & acadabrabarbara\$ \\
      $6$  & $5$     & c        & adabrabarbara\$ \\
      $7$  & $15$    & b        & ara\$ \\
      $8$  & $12$    & b        & arbara\$ \\
      $9$  & $14$    & r        & bara\$ \\
      $10$ & $11$    & a        & barbara\$ \\
      $11$ & $8$     & a        & brabarbara\$ \\
      $12$ & $1$     & a        & bracadabrabarbara\$ \\
      $13$ & $4$     & a        & cadabrabarbara\$ \\
      $14$ & $6$     & a        & dabrabarbara\$ \\
      $15$ & $16$    & a        & ra\$ \\
      $16$ & $9$     & b        & rabarbara\$ \\
      $17$ & $2$     & b        & racadabrabarbara\$ \\
      $18$ & $13$    & a        & rbara\$ \\
      \bottomrule
    \end{tabular}
    \caption{The suffix array for "abracadabrabarbara\$".}
    \label{tbl:burrowsWheelerTransformationExample}
  \end{table}
\end{Example}

\begin{Theorem}
  We can search for all occurrences of a string $S$ in $T$ using the suffix array $SA$ and Burrwos-Wheeler-Transform $BWT$ over $T$ in time $\mathcal{O}(m\log \sigma)$, where $m = \vert S \vert$. This is knows as \defi{backward search}{Backward Search}.
\end{Theorem}

\begin{Proof}
  Bla blub.
\end{Proof}
