\section{Suffix Array}

\begin{Definition}
  The \defi{suffix array}{Suffix Array} \id{SA} of a string $T$ gives the sorted order of all suffixes of $T$. Element $\id{SA}[i]$ stores the index of the $i$-th lexicographically smallest suffix of $T$.
\end{Definition}

\begin{Example}
  Consider the string $T=\texttt{abracadabrabarbara\$}$. Table~\ref{tbl:suffixArrayExample} shows the suffix array for~$T$. Note that the dollar sign \texttt{\$} is lexicographically smaller than any other character. This guarantees that suffixes which are a prefix of other suffixes appear first.
  \begin{table}[htbp]
    \centering
    \begin{tabular}{ccl}
      \toprule
      $i$  & $\id{SA}[i]$ & $T[\id{SA}[i]..n-1]$ \\
      \midrule
      $0$  & $18$         & \texttt{\$} \\
      $1$  & $17$         & \texttt{a\$} \\
      $2$  & $10$         & \texttt{abarbara\$} \\
      $3$  & $7$          & \texttt{abrabarbara\$} \\
      $4$  & $0$          & \texttt{abracadabrabarbara\$} \\
      $5$  & $3$          & \texttt{acadabrabarbara\$} \\
      $6$  & $5$          & \texttt{adabrabarbara\$} \\
      $7$  & $15$         & \texttt{ara\$} \\
      $8$  & $12$         & \texttt{arbara\$} \\
      $9$  & $14$         & \texttt{bara\$} \\
      $10$ & $11$         & \texttt{barbara\$} \\
      $11$ & $8$          & \texttt{brabarbara\$} \\
      $12$ & $1$          & \texttt{bracadabrabarbara\$} \\
      $13$ & $4$          & \texttt{cadabrabarbara\$} \\
      $14$ & $6$          & \texttt{dabrabarbara\$} \\
      $15$ & $16$         & \texttt{ra\$} \\
      $16$ & $9$          & \texttt{rabarbara\$} \\
      $17$ & $2$          & \texttt{racadabrabarbara\$} \\
      $18$ & $13$         & \texttt{rbara\$} \\
      \bottomrule
    \end{tabular}
    \caption{The suffix array for \texttt{abracadabrabarbara\$}.}
    \label{tbl:suffixArrayExample}
  \end{table}
\end{Example}

\begin{Theorem}
  We can search for all occurrences of a string $S$ in $T$ using the suffix array \id{SA} over $T$ in time $\mathcal{O}(m\log n)$, where $n = \vert T \vert$ and $m = \vert S \vert$. This is knows as \defi{forward search}{Forward Search}.
\end{Theorem}

\begin{Proof}
  We find the first suffix in \id{SA} greater or equal to $S$ and the first suffix in $T$ bigger than $S$. Assume these are at positions $i$ and $j$ with $i \leq j$. Then we have $j - i$ occurrences of $S$ in $T$ and the corresponding positions are stored in $\id{SA}[i], \ldots, \id{SA}[j-1]$. Both $i$ and~$j$ can be found using a binary search, since the suffix array is sorted. The binary search needs $\mathcal{O}(\log n)$ steps and each steps does a string comparison in $\mathcal{O}(m)$ time.
\end{Proof}

\begin{Example}
  Consider strings $T=\texttt{abracadabrabarbara\$}$ and $S=\texttt{bar}$. How often and where does~$S$ appear in $T$?

  Table~\ref{tbl:suffixArrayExample} shows the suffix array for $T$. By binary search we find $i = 9$ and $j = 11$. So~$T[\id{SA}[i] \twodots n-1]$ is the smallest suffix greater or equal than \texttt{bar} and $T[\id{SA}[j] \twodots n-1]$ is the smallest suffix greater than \texttt{bar}. We see that \texttt{bar} appears $j-i=11-9=2$ times. The occurrences are at positions $\id{SA}[9]=14$ and $\id{SA}[10]=11$.
\end{Example}

The suffix array itself needs $n\log n$ bits of space. But to work with it, we additionally need to store the text itself, which needs another $n\log\sigma$ bits of space. In total this results in a space of $n\log n + n\log\sigma$ bits.
