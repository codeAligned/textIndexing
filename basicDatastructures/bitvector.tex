\section{Bitvectors}

\begin{Definition}
  A \defi{bitvector}{Bitvector} $B$ is an array of bits that are compactly stored. A bitvector supports the following operations:
  \begin{itemize}
    \item $\proc{Access}(i)$ returns the $i$-th element in $B$.
    \item $\proc{Rank}_q(i)$ returns the number of $q$-bits in the prefix $[0..i-1]$ of $B$. If $q$ is omitted, we assume $q = 1$.
    \item $\proc{Select}_q(i)$ returns the position of the $i$-th $q$-bit. If $q$ is omitted, we assume $q = 1$.
  \end{itemize}
\end{Definition}

\begin{Theorem}
  \label{thm:bitvectorRank}
  The $\proc{Rank}$-Operation on bitvectors can be done constant time and $o(n)$ additional space.\cite{Jacobson1989}
\end{Theorem}

\begin{Proof}
  Let $B$ be a bitvector of length $n$. Precompute the following information:
  \begin{enumerate}
    \item Divide $B$ into \emph{superblocks} $SB_1, \ldots, SB_{\lceil \frac{n}{L} \rceil}$ of size $L$.
    \item For each superblock $SB_j$ now store $\sum_{i=0}^{(j-1)L-1} B[i]$. This is the number of set bits in the first $j$ superblocks. For each superblock this needs $\mathcal{O}(\log n)$ bits of space.
    \item Now further divide each superblock $B_j$ into blocks $B_1^j, \ldots, B_{\lceil \frac{L}{S} \rceil}^j$ of size $S$.
    \item For each block $B_k^j$ of superblock $SB_j$ store $\sum_{i=(j-1)L}^{(j-1)L+kS-1} B[i]$. This is the number of set bits in the first $k$ blocks of superblock $SB_j$. For each block this needs $\mathcal{O}(\log L)$ bits.
  \end{enumerate}
  For a given $\proc{Rank}(i)$ query we now calculate the corresponding superblock $SB_j$ and block $B_k$. We use the precomputed sums to get the number of set bits in all superblocks before $SB_j$ and all blocks before $B_k$. All thats missing now is the number of set bits in block $B_k$ until position $i$. If the blocks are small enough, this information can be precomputed efficiently for all possible blocks and positions (Four-Russians-Trick).

  We still need to choose $L$ and $S$:
  \begin{align}
    L &= \log^2 n \\
    S &= \frac{1}{2} \log n
  \end{align}

  Let's look at the space needed:
  \begin{itemize}
    \item Prefix sums among the superblocks:
    \begin{align}
      \begin{aligned}
        \mathcal{O}\left(\frac{n}{L}\log n\right)
        &= \mathcal{O}\left(\frac{n}{\log^2 n}\log n\right)\\
        &= \mathcal{O}\left(\frac{n}{\log n}\right)
      \end{aligned}
    \end{align}
    \item Prefix sums among the blocks:
    \begin{align}
      \begin{aligned}
        \mathcal{O}\left(\frac{n}{S}\log L\right)
        &= \mathcal{O}\left(\frac{n}{\log n}\log\log^2 n\right) \\
        &= \mathcal{O}\left(\frac{n}{\log n} \cdot \log\log n\right) \\
        &= \mathcal{O}\left(\frac{n \log\log n}{\log n}\right)
      \end{aligned}
    \end{align}
    \item The lookup tables for each block. There are $2^S$ possible blocks and for each of them we need to store $S$ prefix sums in $\mathcal{O}(\log S)$ bits:
    \begin{align}
      \begin{aligned}
        \mathcal{O}\left(2^S \cdot S \cdot \log S\right)
        &= \mathcal{O}\left(2^{\frac{1}{2}\log n} \cdot \frac{1}{2}\log n \cdot \log \frac{1}{2}\log n\right) \\
        &= \mathcal{O}\left(\sqrt{n}\log n \log\log n\right)
      \end{aligned}
    \end{align}
  \end{itemize}

  The total amount of space needed is therefore $\mathcal{O}\left(\frac{n}{\log n} + \frac{n\log\log n}{\log n} + \sqrt{n}\log n\log\log n\right) = o(n)$ bits.
\end{Proof}
