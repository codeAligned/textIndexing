\chapter{Introduction}

\begin{Definition}
  Let $\Sigma = \{0, \ldots, \sigma - 1\}$ be a finite, ordered set. The elements of $\Sigma$ are called \defi{characters} or \defi{symbols} and $\Sigma$ is called an \defi{alphabet} of size $\sigma$.
\end{Definition}

\begin{Definition}
  A \defi{string} $S$ is a sequence of characters from an alphabet $\Sigma$.
  \begin{itemize}
    \item We usually use $n = \vert S \vert$ to be the length of the string.
    \item The $i$-th character of $S$ is $S[i]$. Indices are $0$-based.
    \item The substring from the $i$-th to the $j$-th character is $S[i..j]$.
    \item A substring with $i = 0$ is called \defi{prefix}. A substring with $j = n - 1$ is called \defi{suffix}.
    \item The \defi{$i$-th suffix} is $S[i..n-1]$.
  \end{itemize}
\end{Definition}

\section{Tries}

\begin{Definition}
  Let $S = \{S_0, S_1, \ldots, S_{N-1}\}$ be a set of strings over an alphabet $\Sigma$. A \defi{trie} is a tree, where each node represents a different prefix in the set $S$. The root represents the empty prefix $\varepsilon$. Vertex $u$ representing prefix $Y$ is a child of vertex $v$ representing prefix $X$, if and only if $Y = Xc$ for some character $c \in \Sigma$. The edge $(v,u)$ is then labeled $c$.

  If $S$ is the set of all suffixes of a string $T$, the trie is called \defi{suffix trie}.
\end{Definition}

\begin{Example}
  Figure~\ref{fig:trieExample} shows the suffix trie for the string \"banana\$\". The dollar sign \$ is a sentinal that does not appear elsewhere in the text. This guarantees, that no suffix is a prefix of another suffix and the suffix trie therefore has $n+1$ leaves.
  \drawing{introduction/trieExample.tex}{Hallo}{trieExample}
\end{Example}
